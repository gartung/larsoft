
\chapter{ Capabilities }
\label{ch:capabilities}
Add overarching  usability and efficiency.
\begin {itemize}
\item Raw event content and handling

The TPC DAQ systems themselves are outside of the scope of the implementation plan. The needs are to ensure that, as stakeholders to the DAQ, the offline and computing systems receive the event data needed in the format, time, and with the metadata needed. 
All known requirements are currently met by the general artDAQ system with the exception of the following:
\begin{itemize} 
\item  ability of the DAQ system to provide streaming output of event data that can be split according to varying needs of the offline system after acquisition.
\item ... performance for later DUNE needs
\item ... ease of integration into.. 
\end {itemize}


\item Calibration data - 

how much is in common and can be specified as part of the larsoft development? 

\item Conditions data

how much is in common and can be specified as part of the larsoft development? 

\item Signal processing



\item Hit-finding
\item Cluster-finding
\item Track finding
\item Shower reconstruction
\item Vertex finding
\item Event time (t0)
\item Particle identification
\item General reconstruction
\item Event-level reconstruction
\item Simulation output and data structures
\item General simulation features and capabilities
\item Random number seeds
\item Reconstruction and simulation
\item Dual-phase LArTPC simulation
\item Documentation of data structures and algorithms
\item Documentation of the environments
\item Data files
\item Data objects and event data model
\item Analysis processing
\item Visualization for debugging and tuning
\item Interactive capabilities of the visualization
\item Data distribution and preservation
\item Metadata management
\item Software build tools
\item Software environment configuration tools
\item Database support
\item Analysis workflow support
\item Computing systems and related support
\item Organizational structures and processes
\end {itemize}
