%\documentclass[draftmode,draftwater]{memarticle}
\documentclass{memarticle}
%\documentclass[screen]{memarticle}

% ------------------------------------------------------------
% Document title control. Change these as appropriate for your
% document. la la la
% ------------------------------------------------------------
%\newcommand{\doctitle}{Requirements for a shared LArTPC offline software suite}
\newcommand{\doctitle}{LArTPC Community for Offline Software and  Computing}
\newcommand{\brieftitle}{LArTPC Implementation Plan}
\newcommand{\authors}{\href{mailto:larsoft-coordination@fnal.gov}{Lots of names here}}

%%%%%%%%%%%%%%%%%%%%%%%%%%%%%%%%%%%%%%
%%%%                                                                                                %%%%
%%%%   The document version and date completed goes here!!!     %%%%
%%%%   in the \docversion command                                                %%%%
%%%%                                                                                                %%%%
\newcommand{\docversion}{0.1}
%
%%%%%%%%%%%%%%%%%%%%%%%%%%%%%%%%%%%%%%


\newlength{\mylen}% New length to use when setting indentation

\settowidth{\mylen}{\large\bfseries Contents}% Sets mylen to width of word “Contents”, large and bold
\cftsetindents{subsection}{0pt}{\mylen}% Sets indent and numwidth for sections
\renewcommand{\cftsubsectionpresnum}{\hfill}
\renewcommand{\cftsubsectionaftersnum}{\hfill}
\renewcommand{\cftsubsectionaftersnumb}{\hspace{-1em}}
\renewcommand*\contentsname{}

% ------------------------------------------------------------
% Define new commands here
% ------------------------------------------------------------
\newcommand{\vs}{\textit{vs.}}
\newcommand{\maxproducts}{100\xspace}
\newcommand{\nb}{\textit{N.B.}\xspace}

%\newcommand{\productname}[1]{\textbf{\textcolor{cyan}{#1}}\xspace}
\newcommand{\productname}[1]{\textit{{#1}}\xspace}
\newcommand{\Rprod}{\productname{R}}
\newcommand{\Sprod}{\productname{S}}
\newcommand{\artdaq}{\productname{artdaq}}
\newcommand{\artgf}{\productname{artG4}}
\newcommand{\art}{\productname{art}}
\newcommand{\boost}{\productname{Boost}}
\newcommand{\cetlib}{\productname{cetlib}}
\newcommand{\cling}{\productname{Cling}}
\newcommand{\cmake}{\productname{CMake}}
\newcommand{\cmssw}{\productname{CMSSW}}
\newcommand{\cstxsd}{\productname{CodeSynthesis XSD}}
\newcommand{\cvmfs}{\productname{CVMFS}}
\newcommand{\fhicl}{\productname{FHiCL}}
\newcommand{\fhiclcpp}{\productname{fhicl-cpp}}
\newcommand{\FIFE}{\productname{FIFE}}
\newcommand{\fortran}{Fortran\xspace}
\newcommand{\gcc}{\productname{GCC}}
\newcommand{\gccxml}{\productname{GCC-XML}}
\newcommand{\geant}{\productname{Geant4}}
\newcommand{\git}{\productname{Git}}
\newcommand{\ifbeam}{\productname{ifbeam}}
\newcommand{\ifdh}{\productname{IFDH}}
\newcommand{\larsoft}{\productname{LArSoft}}
\newcommand{\mf}{\productname{messagefacility}}
%\newcommand{\mpi}{\productname{MPI}}
\newcommand{\mpich}{\productname{MPICH}}
\newcommand{\mvapich}{\productname{MVAPICH}}
\newcommand{\mrb}{\productname{MRB}}
\newcommand{\nutools}{\productname{Nutools}}
\newcommand{\PANDORA}{\productname{PANDORA}}
\newcommand{\python}{\productname{Python}}
\newcommand{\redmine}{\productname{Redmine}}
\newcommand{\roofit}{\productname{RooFit}}
\newcommand{\rootprod}{\productname{ROOT}}
\newcommand{\ruby}{\productname{Ruby}}
\newcommand{\rups}{\productname{relocatable UPS}}
\newcommand{\scons}{\productname{SCONS}}
\newcommand{\sigcpp}{\productname{SIG\cpp}}
\newcommand{\sqlite}{\productname{SQLite}}
\newcommand{\srt}{\productname{SoftRelTools}}
\newcommand{\sam}{\productname{SAM}}
\newcommand{\smc}{\productname{SMC}}
\newcommand{\svn}{\productname{Subversion}}
\newcommand{\swig}{\productname{swig}}
\newcommand{\tbb}{\productname{Intel TBB}}
\newcommand{\ups}{\productname{UPS}}
\newcommand{\worch}{\productname{worch}}
\newcommand{\xercesc}{\productname{xerces-c}}

%\newcommand{\expt}[1]{\textbf{\textcolor{Goldenrod}{#1}}\xspace}
\newcommand{\expt}[1]{\textbf{{#1}}\xspace}
\newcommand{\argoneut}{\expt{ArgoNeuT}}
\newcommand{\cms}{\expt{CMS}}
\newcommand{\dsf}{\expt{DS50}}
\newcommand{\ds}{\expt{Darkside}}
\newcommand{\dzero}{\expt{D\O}}
\newcommand{\gm}{\expt{Muon g-2}}
\newcommand{\lbne}{\expt{LBNE}}
\newcommand{\muboone}{\expt{$\mu$BooNE}}
\newcommand{\mue}{\expt{Mu2e}}
\newcommand{\nova}{\expt{NO$\nu$A}}
\newcommand{\cosmosis}{\expt{CosmoSIS}}

\newcommand{\A}[1]{#1\xspace}
\newcommand{\Q}[1]{\textcolor{green}{#1}\xspace}
\newcommand{\blankline}{\vskip\baselineskip}

\usepackage{nameref}

% Package for and list of authors
\usepackage{authblk}
\makeatletter
\renewcommand\AB@affilsepx{, \protect\Affilfont}
\makeatother
\author {LArTPC Workshop Organizing Group}
%%\author[26]{C.~Adams}
%\author[7]{J.~Amundson}
%\author[25]{J.~Asaadi}
%\author[7]{B.~Baller}
%\author[11]{T.~Bolton}
%\author[7]{S.~Brice}
%\author[7]{F.~Cavanna}
%\author[4]{D.~Cherdack}
%\author[16]{E.~Church}
%\author[25]{A.~Farbin}
%\author[9]{C.~Farnese}
%\author[26]{B.~Fleming}
%\author[19 ]{W.~Foreman}
%\author[8]{V.~Galymov}
%\author[22]{D.~Garcia~Gamez}
%\author[9]{D.~Gibin}
%\author[7]{H.~Greenlee}
%\author[15]{R.~Guenette}
%\author[12]{T.~Hasegawa}
%\author[19 ]{J.~Ho}
%\author[7]{C.~Jones}
%\author[7]{T.~Junk}
%\author[7]{W.~Ketchum}
%\author[23]{J.~Klein}
%\author[7]{J.~Kowalkowski}
%\author[7]{P.~Kryczynski}
%\author[7]{R.~Kutschke}
%\author[14]{T.~Kutter}
%\author[7]{A.~Lyon}
%\author[15]{J.~Marshall}
%\author[3]{S.~Murphy}
%\author[24]{D.~Naples}
%\author[22]{J.~Nowak}
%\author[7]{O.~Palamara}
%\author[7]{M.~Paterno}
%\author[8]{E.~Pennacchio}
%\author[7]{G.~Petrillo}
%\author[7]{R.~Pordes}
%\author[1]{X.~Qian}
%\author[7]{J.~Raaf}
%\author[7]{B.~Rebel}
%\author[7]{R.~Roser}
%\author[6]{A.~Rubbia}
%\author[3]{P.~Sala}
%\author[19 ]{D.~Schmitz}
%\author[7]{S.~Sehrish}
%\author[7]{E.~Snider}
%\author[7]{P.~Spentzouris}
%\author[20 ]{J.~St.John}
%\author[7]{M.~Stancari}
%\author[3]{D.~Stefan}
%\author[3]{R.~Sulej}
%\author[22]{A.~Szelc}
%\author[5]{K.~Terao}
%\author[15]{M.~Thomson}
%\author[ ]{M.~Torti}
%\author[13]{C.~Tull}
%\author[17]{T.~Usher}
%\author[1]{B.~Viren}
%\author[20]{Lisa~Whitehead}
%\author[1]{E.~Worcester}
%\author[7]{T.~Yang}
%\author[26]{G.~Zeller}
%\author[19 ]{J.~Zennamo}

%%\affil[1]{Brookhaven~National~Lab}
%\affil[2]{Cambridge~University}
%\affil[3]{CERN}
%\affil[4]{Colorado~State~University}
%\affil[5]{Columbia~University}
%\affil[6]{ETHZ}
%\affil[7]{Fermilab}
%\affil[8]{IN2P3}
%\affil[9]{INFN~Padua}
%\affil[10]{INFN~Pavia}
%\affil[11]{Kansas~State~University}
%\affil[12]{KEK}
%\affil[13]{LBL}
%\affil[14]{Louisiana~State~University}
%\affil[15]{Oxford~University}
%\affil[16]{PNNL}
%\affil[17]{SLAC}
%\affil[18]{Syracuse~University}
%\affil[19]{University~of~Chicago}
%\affil[20]{University~of~Cincinnati}
%\affil[20]{University~of~Houston}
%\affil[[21]{University~of~Lancaster}
%\affil[22]{University~of~Manchester}
%\affil[23]{University~of~Pennsylvania}
%\affil[24]{University~of~Pittsburgh}
%\affil[25]{University~of~Texas~Austin}
%\affil[26]{Yale~University}



% Control the space for chapter/section/subsection numbers in the Table
% of Contents. The default doesn't leave enough room for long labels.
\cftsetindents{chapter}{0em}{5em}
\cftsetindents{section}{0em}{5em}
\cftsetindents{subsection}{0em}{5em}

\usepackage{xcolor}
\usepackage{datetime}
\usepackage{siunitx}
\newdate{date}{06}{09}{2012}


%\usepackage[titletoc,title]{appendix}

%------------------------------------------------------------
% Document starts here
%------------------------------------------------------------
\begin{document}
%\currenttime
\date{2016}
\maxtocdepth{subsection}
\maxsecnumdepth{paragraph}


%% replace with centered and author list \topmatter	
\title{\doctitle}
\maketitle
\begin{abstract}
 
Experiments, using or planning to use a Liquid Argon Time Projection Chamber as part of their detector systems, are 
working together  towards  sharing components of their offline data simulation, processing and analysis solutions. 
This document gives the high level plan to meet the requirements and goals gathered  from the community of 
experiments, projects and organizations for software and computing components in the last quarter of 2015.
 
\end{abstract}
\newpage
\tableofcontents

\newpage
% this is a comment
% this is another comment
% this is a third comment

%------------------------------------------------------------

\chapter {Introduction}

The systems being implemented through this plan include the  offline software and computing to meet the physics requirements of  liquid argon-based experiments -  including data simulation, reconstruction and analysis systems. The systems that are being, and will in the future be, implemented  are ultimately the experiment data
simulation, reconstruction and analysis systems themselves, specific
to each collaboration. 

The following principles are being followed to determine the scope, timeline and priority of the implementations:
\begin {itemize}
\item  The Stakeholders determine the ultimate deliverables of a project.
\item Stakeholders can �demand� an end-date.
\item The implementation includes a distributed set people, software, systems. 
\end {itemize}

The  Stakeholders for this implementation are:  
\begin {itemize}
\item  Experiments
\item  Individual Users
\item  Team members
\item   Overseers
\end {itemize}

The Requirements that drive this implementation are given by {ref}. This implementation document includes a discussion of how to determine if a particular requirements is met. 
 
\section {Process}
Planning Steps happen at the top level and iteratively at successful levels of detail with looping back to the top level and for each of the details.
Loop for initial development and additions/modification of capability, breaking of backward compatibility:
�       Requirements
�       Top level is iterations of the requirements document
�       Definition and architecture of the end-to-end deliverable
�       Top level is the the architecture document (including workflows) until determine is not sufficient.
�       Organizational infrastructure, responsibilities and decision making
�      Definition and architecture of sub-systems/activities
Loop for continuous small improvements:
�       Evaluation and Feasibility studies
�       Development of infrastructure and sub-systems (including hardware, people, workflows, software)
�       Integration, unit test and release
�       End to end tests and data challenges
�       Deployment
�       Maintenance, operations and support
End of both loops.
 
De-scoping needs: Stakeholders who understand the impact (both science goals and technology needs) in the loop at the right time.
 
We propose to plan to the following processes (methods) Methodologies:

\begin {itemize}
\item Rolling Wave Planning: project planning in waves as the project proceeds and later details become clearer. Work to be done in the near term is based on high level assumptions; also, high level milestones are set. As the project progresses, the risks, assumptions, and milestones originally identified become more defined and reliable. One would use Rolling Wave Planning in an instance where there is an extremely tight schedule or timeline to adhere to; whereas more thorough planning would have placed the schedule into an unacceptable negative schedule variance.

\item Agile development: in which requirements and solutions evolve through collaboration between self-organizing, cross-functional teams. It promotes adaptive planning, evolutionary development, early delivery, continuous improvement, and encourages rapid and flexible response to change.
Use git terminology for releases � tags, branches, versions, releases.
\end {itemize}

\section {Goals}
Agreed on approach to planning through a) definition of high level Goals  - timelines and deliverables - for each experiment; b) identifying what is needed for each capability in the Requirements document for each Milestone; c) identifying gaps and revisit Requirements document; d) drilling down for each capability on more specifics; e) recording as plan and resources.

\subsection {DUNE 35ton}

\begin{description}
\item[3/1/2016] reconstruct straight tracks in the detector. 
\item[6/1/2016] complete operations.
\item[1/1/2017]  publish papers � lifetime, geometry tests, studies on showers.
\end{description}
 
\subsection {ProtoDUNE}
\begin{description}
\item[1/1/2018] data taking starts.
\item[1/12/2018] data taking completes.
\item[2019] analyze data and publish results.
\end{description}
 
\subsection {DUNE}
\begin{description}
\item[1/1/2016] Decision on whether to use handscans as well as automated reconstruction for conclusions of optimization TBA 
\item[1/31/2016] Dual phase simulation.
\item[3/1/2016]  results from FD and ND task force simulations with as much of the reconstruction capability working.
\item[6/1/2016]  Some working full chain of the reconstruction.
\item[6/1/2016]  Detector optimization for ProtoDUNE decisions.
\item[6/1/2016]   Interactive visualization that feeds back to LArSoft � Bee � version that allows handscan that can feed back into LArSoft to deliver to optimization conclusions � TBA 
\item[1/1/2017]  Conclusions of optimization.
\item[2018/2019] Decide on path for framework, infrastructure and toolkit for 2025 framework.
\item[2021] Detector installation starts 2021.
\item[2025] Data taking. 
\end{description}

\subsection {Lariat}
\begin{description}
\item[1/1/2016] Upgrade Geant4 version for Kaon lifetime analysis 4.10 
\item[1/31/2016] Support for Pion, Kaon, and lifetime publications.
\item[9/1/2016] Exclusive cross section measurements for pion interactions, charge exchange,mu anti-mu analysis, michel electrons, pi-zero analysis 
\item[FY17] Continued analysis and paper writing.
\item[7/1/2016] Decision on further running.
\item[2015-2017]  Support for publications.
\end{description}


\subsection {MicroBooNE}
\begin{description}
\item[3/1/2016] Cosmic rate paper, diffusion measurement and purity paper.
\item[4/1/2016] Cosmic tagging sufficient for cross section measurement. 
\item[6/1/2016]    Muon inclusive cross-section measurement.
\item[1/1/2017]    Oscillations group output and included TBA .
\item[6/1/2017]     CCQE, Pion production, neutral current/PIzero cross-section papers.
\item[6/1/2017]  Publication of Pizero reconstructed mass peak, gamma separation.
\item[ 7/1/2017] Support for the new cosmic veto TBA.
\item[ Post-2017] -Oscillation measurement TBA.
\item[ 10/1/2018]   Support for 6.6 e**20 POTs, as SBN  Microboone TBA 
\end{description}
 
\subsection {SBND} 
\begin{description}
\item[6/1/2016] Full reconstruction chain.
\item[10/1/2016] MC data challenge 1 with full reconstruction, Neutrino and cosmic ray
\item[3/1/2017] MC data challenge 2 with full reconstruction and automated analysis.
\item[6/1/2018] Ready for reconstruction of data for commissioning TBA.
\end{description}


\chapter{ Capabilities }
\label{ch:capabilities}
Add overarching  usability and efficiency.
\begin {itemize}
\item Raw event content and handling

The TPC DAQ systems themselves are outside of the scope of the implementation plan. The needs are to ensure that, as stakeholders to the DAQ, the offline and computing systems receive the event data needed in the format, time, and with the metadata needed. 
All known requirements are currently met by the general artDAQ system with the exception of the following:
\begin{itemize} 
\item  ability of the DAQ system to provide streaming output of event data that can be split according to varying needs of the offline system after acquisition.
\item ... performance for later DUNE needs
\item ... ease of integration into.. 
\end {itemize}


\item Calibration data - 

how much is in common and can be specified as part of the larsoft development? 

\item Conditions data

how much is in common and can be specified as part of the larsoft development? 

\item Signal processing



\item Hit-finding
\item Cluster-finding
\item Track finding
\item Shower reconstruction
\item Vertex finding
\item Event time (t0)
\item Particle identification
\item General reconstruction
\item Event-level reconstruction
\item Simulation output and data structures
\item General simulation features and capabilities
\item Random number seeds
\item Reconstruction and simulation
\item Dual-phase LArTPC simulation
\item Documentation of data structures and algorithms
\item Documentation of the environments
\item Data files
\item Data objects and event data model
\item Analysis processing
\item Visualization for debugging and tuning
\item Interactive capabilities of the visualization
\item Data distribution and preservation
\item Metadata management
\item Software build tools
\item Software environment configuration tools
\item Database support
\item Analysis workflow support
\item Computing systems and related support
\item Organizational structures and processes
\end {itemize}


\chapter{Repositories}

\begin{itemize}
\item https://cdcvs.fnal.gov/redmine/projects/larsoft
\item https://github.com/DUNE
\end{itemize}
%\bibliographystyle{apalike}
%\bibliography{Refs}

%\appendix
% \renewcommand{\chaptername}{Appendix}


\end{document}

%%% Local Variables:
%%% mode: latex
%%% TeX-master: t
%%% End: